\documentclass[10pt, oneside]{article}
\usepackage{amsmath, amsthm, amssymb, calrsfs, wasysym, verbatim, bbm, color, graphics, geometry}

\geometry{tmargin=.50in, bmargin=.50in, lmargin=.50in, rmargin = .50in}  

\newcommand{\R}{\mathbb{R}}
\newcommand{\C}{\mathbb{C}}
\newcommand{\Z}{\mathbb{Z}}
\newcommand{\N}{\mathbb{N}}
\newcommand{\Q}{\mathbb{Q}}
\newcommand{\Cdot}{\boldsymbol{\cdot}}

\newtheorem{theorem}{Theorem} % 定理
\newtheorem{definition}{Definition} % 定义
\newtheorem{property}{Property} % 性质
\newtheorem{convention}{Convention} % 约定
\newtheorem{remark}{Remark} % 注释
\newtheorem{lemma}{Lemma} % 引理
\newtheorem{corollary}{Corollary} % 推论


\title{SPH Method}
\author{bcynuaa}
\date{\today}

\begin{document}

\maketitle
\tableofcontents

\vspace{.25in}

\section{Math Basics}

\subsection{Kernal Function}

In SPH method, we use kernal function to approximate $\delta$ function. 
The kernal function is defined as follows:

\begin{definition}
    A function $W: \R^d \to \R$ is called a kernal function if it satisfies the following conditions:
    \begin{enumerate}
        \item $W$ is a continuous function.
        \item $W$ is a radially symmetric function, i.e. $W(x) = W(y) \quad \forall |x|=|y|, x, y \in \R^d$.
        \item $W$ is a positive definite function, i.e. $\int_{\R^d} W(x) dx = 1$.
    \end{enumerate}
\end{definition}

\begin{remark}
    The kernal function is usually chosen as a compactly supported function, i.e. $W(x) = 0$ for $|x| > h$.
\end{remark}

Kernal functions's formation is not unique. 
Theses series functions have property as follows:

\begin{property}
    % 筛选性
    Screening property.
    \begin{equation}
        \int_{\R^d} f(x^\prime)W(x-x^\prime) dx^\prime = f(x)
    \end{equation}
\end{property}

\begin{property}
    % 紧致性
    Compactness.
    \begin{equation}
        W(x)=0 \quad \forall |x| > h
    \end{equation}
\end{property}

\begin{property}
    % 对称性
    Symmetry.
    \begin{equation}
        \begin{aligned}
            W(x) &= W(-x)\\
            -\nabla W(x) &= \nabla W(-x)
        \end{aligned}
    \end{equation}
\end{property}

Here $h$ is called the smoothing length. 
It's a parameter to control the compactness of the kernal function.

\subsection{Particle Approximation}

In SPH method, we use particle approximation to approximate the integral of a function. 
Let's consider a function $u: \R^d \to \R$ and a point $x \in \R^d$.
$u(x)$ can be approximated by the following equation:

\begin{equation}
    u(x) = \int_{\R^d} u(x^\prime) W(x-x^\prime) dx^\prime
\end{equation}

Consider a particle system $\{x_i\}_{i=1}^N$ in $\R^d$. 
Particles in this system are distributed all over the space. 
Each particle has a mass $m_i$ and a density $\rho_i$, 
and other physical properties such as velocity, pressure, etc. 
$u$ represents a physical property of the particle system.

Each particle captures a small region of the space. 
Only particles in this region have significant contribution to the integral.
Thus we could approximate the integral by summing up the contribution of all particles in the system.
Consider scalar $u$'s value at particle $x_i$ is $u_i$, 
then the approximation of $u(x)$ is as follows:

\begin{equation}
    \begin{aligned}
        u_i &= \int_{\R^d} u(x^\prime) W(x_i-x^\prime) dx^\prime\\
        &= \sum_{j=1}^{N} \Delta V_j u_j W(x_i-x_j)
    \end{aligned}
\end{equation}

Here $\Delta V_j$ is the volume of particle $x_j$, 
which can be calculated by $\Delta V_j = m_j / \rho_j$. 
Denote $W_{ij} = W(x_i-x_j)$, then we have:

\begin{equation}
    u_i = \sum_{j=1}^{N_i} \frac{m_j}{\rho_j} u_j W_{ij}
    = \sum_{j=1}^{N_i} \frac{m_j}{\rho_j} u_j W_{ji}
\end{equation}

Similarly, 
we can approximate the gradient of $u$ by the following equation:

\begin{equation}
    \begin{aligned}
        \nabla u 
        &= \int_{\R^d} \nabla u(x^\prime) W(x-x^\prime) dx^\prime\\
        &\mathop{=}^{\text{symmetric}} \int_{\R^d} \nabla u(x^\prime) W(x^\prime-x) dx^\prime\\
        &=\int_{\partial \Omega} u(x^\prime) \vec{n} W(x^\prime-x) dS^\prime - 
        \int_{\R^d} u(x^\prime) \nabla W(x^\prime-x) dx^\prime\\
        &\mathop{=}^\text{compactness}-\int_{\R^d} u(x^\prime) \nabla W(x^\prime-x) dx^\prime
    \end{aligned}
\end{equation}

Denote $\nabla u$'s $k$-th component as $u_{,k}$, $\nabla W$'s $k$-th component as $W_{,k}$, 
then we have:

\begin{equation}
    u_{i,k} = -\sum_{j=1}^{N_i} \frac{m_j}{\rho_j} u_j W_{ji,k}
    = \sum_{j=1}^{N_i} \frac{m_j}{\rho_j} u_j W_{ij,k}
\end{equation}

Here $W_{ji,k}$ is:

\begin{equation}
    W_{ji,k} = \frac{\partial W(x)}{\partial x_{k}}|_{x=x_j-x_i}
    =-W_{ij,k}
\end{equation}

For particular work, Danis proposed a more widely used approximation of gradient:

\begin{equation}
    \begin{aligned}
        u_{i,k}&=\frac{1}{\rho_i}\sum_{j=1}^{N_i} m_j (u_j-u_i) W_{ij,k}\\
        u_{i,k}&=\rho_i \sum_{j=1}^{N_i} m_j
        \left(
            \frac{u_i}{\rho_i^2} + \frac{u_j}{\rho_j^2}
        \right) W_{ij,k}
    \end{aligned}
\end{equation}

The detailed formula can be seen in Dan Koschier's paper.

Laplace operator can be approximatedas below:

\begin{equation}
    \begin{aligned}
        \nabla^2 u &= \int_{\R^d} \nabla^2 u(x^\prime) W(x-x^\prime) dx^\prime\\
        &= \int_{\R^d} \nabla \cdot \nabla u(x^\prime) W(x^\prime - x) dx^\prime\\
        &= \int_{\partial \Omega} \nabla u(x^\prime) \cdot \vec{n} W(x^\prime - x) dS^\prime -
        \int_{\R^d} \nabla u(x^\prime) \cdot \nabla W(x^\prime - x) dx^\prime\\
        &\mathop{=}^\text{compactness} -\int_{\R^d} \nabla u(x^\prime) \cdot \nabla W(x^\prime - x) dx^\prime\\
        &= -\int_{\R^d} u(x^\prime)\vec{n} \cdot \nabla W(x^\prime - x) dx^\prime +
        \int_{\R^d} u(x^\prime) \nabla \cdot \nabla W(x^\prime - x) dx^\prime\\
        &\mathop{=}^\text{compactness} \int_{\R^d} u(x^\prime) \nabla^2 W(x - x^\prime) dx^\prime\\
        &\approx \sum_{j=1}^{N_i} \frac{m_j}{\rho_j} u_j \nabla^2 W_{ij}
    \end{aligned}
\end{equation}

However, according to Dan Koschier's paper, 
this form of approximation is not stable and accurate enough. 
A imporved discrete Laplacia operator is proesented by Espanol and Revenga: 

\begin{equation}
    \nabla^2 u  = \sum_{j=1}^{N_i} \frac{2m_j}{\rho_j}
    (u_j-u_i) \frac{x_{ij}\cdot \nabla W_{ij}}{|x_{ij}|^2 + \eta^2}
\end{equation}

where $x_{ij} = x_i - x_j$ and $\eta$ is a small positive constant which 
usually set to be $0.01h$ to prevent singularity.

Now considering a vector function $\vec{v}: \R^d \to \R^d$, 
we can approximate the divergence of $\vec{v}$ by the following equation:

\begin{equation}
    \begin{aligned}
        \nabla \cdot \vec{v} &= \int_{\R^d} \nabla \cdot \vec{v}(x^\prime) W(x-x^\prime) dx^\prime\\
        &= \int_{\partial \Omega} \vec{v}(x^\prime) \cdot \vec{n} W(x^\prime - x) dS^\prime -
        \int_{\R^d} \vec{v}(x^\prime) \cdot \nabla W(x^\prime - x) dx^\prime\\
        &\mathop{=}^\text{compactness} -\int_{\R^d} \vec{v}(x^\prime) \cdot \nabla W(x^\prime - x) dx^\prime\\
        &= \int_{\R^d} \vec{v}(x^\prime) \cdot \nabla W(x - x^\prime) dx^\prime\\
    \end{aligned}
\end{equation}

In particle approximation, we have:

\begin{equation}
    (\nabla \cdot \vec{v})_i = \sum_{j=1}^{N_i} \frac{m_j}{\rho_j} \vec{v}_j \cdot \nabla W_{ij}
\end{equation}

\section{Formula of Kernal Function}

Usually, kernal function can be seen as a function of $q$:

\begin{equation}
    W(q) = W\left(\frac{|\vec{r}|}{h}\right)
\end{equation}

Where:

\begin{equation}
    \vec{r} = \vec{x} - \vec{x}^\prime
\end{equation}

denote that:

\begin{equation}
    \begin{aligned}
        \vec{r} = x_j \vec{e}_j \quad r = |\vec{r}|\\
    \end{aligned}
\end{equation}

we will have:

\begin{equation}
    \begin{aligned}
        \frac{dq}{dr} &= \frac{1}{h}\\
        \frac{\partial r}{\partial x_j} &= \frac{x_j}{r}\\
    \end{aligned}
\end{equation}

\subsection{Gradient of Kernal Function}

With einstein notation, we have:

\begin{equation}
    \nabla W(\vec{r}, h) = \frac{\partial W(\vec{r},h)}{\partial x_j}\vec{e}_j
\end{equation}

$\nabla W$'s component is:

\begin{equation}
    \begin{aligned}
        \frac{\partial W}{\partial x_j}&=
        \frac{\mathrm{d} W}{\mathrm{d} q}\frac{\mathrm{d} q}{\mathrm{d} r}\frac{\partial r}{\partial x_j}\\
        &=\frac{\mathrm{d} W}{\mathrm{d} q}\frac{1}{h}\frac{x_j}{r}\\
    \end{aligned}
\end{equation}

Thus we have:

\begin{equation}
    \nabla W(\vec{r}, h) = W^\prime \frac{1}{h} \frac{\vec{r}}{r}
\end{equation}

\subsection{Laplacian of Kernal Function}

With einstein notation, we have:

\begin{equation}
    \nabla^2 W(\vec{r}, h) = \frac{\partial^2 W(\vec{r},h)}{\partial x_j^2}
\end{equation}

$\nabla^2 W$'s component is:

\begin{equation}
    \begin{aligned}
        \frac{\partial^2 W}{\partial x_j^2}&=\frac{\partial}{\partial x_j}
        \left(
            \frac{\mathrm{d} W}{\mathrm{d} q}\frac{1}{h}\frac{x_j}{r}
        \right)\\
        &=\frac{1}{h}\frac{x_j}{r}\frac{\mathrm{d}^2 W}{\mathrm{d} q^2}\frac{1}{h}\frac{x_j}{r}+
        \frac{\mathrm{d} W}{\mathrm{d} q}\frac{1}{h}\frac{\partial}{\partial x_j}\left(\frac{x_j}{r}\right)\\
        &=\frac{1}{h^2}\frac{x_j^2}{r^2}\frac{\mathrm{d}^2 W}{\mathrm{d} q^2}+
        \frac{\mathrm{d} W}{\mathrm{d} q}\frac{1}{h}\left(\frac{1}{r}-\frac{x_j^2}{r^3}\right)\\
        &=\frac{1}{h^2}\frac{x_j^2}{r^2}W^{\prime\prime}+
        W^\prime\frac{1}{h}\frac{r^2-x_j^2}{r^3}\\
    \end{aligned}
\end{equation}

Thus we have:

\begin{equation}
    \begin{aligned}
        \nabla^2 W(\vec{r}, h) &= \frac{1}{h^2}\frac{\vec{r}\cdot\vec{r}}{r^2}W^{\prime\prime}+
        W^\prime\frac{1}{h}\frac{\text{dim}r^2-\vec{r}\cdot\vec{r}}{r^3}\\
        &=\frac{1}{h^2}\frac{r^2}{r^2}W^{\prime\prime}+
        W^\prime\frac{1}{h}\frac{(\text{dim}-1)r^2}{r^3}\\
        &=\frac{1}{h^2}W^{\prime\prime}+
        \frac{(\text{dim}-1)}{hr}W^\prime\\
    \end{aligned}
\end{equation}

notice that, when $r\to 0$, for $W^{\prime}\to 0$, we could have an approximation:

\begin{equation}
    \begin{aligned}
        \frac{\text{dim}-1}{hr}W^\prime &=
        \frac{\text{dim}-1}{h^2\frac{r}{h}}W^\prime\\
        &=\frac{\text{dim}-1}{h^2}\lim_{q\to 0}\frac{W^\prime-0}{q-0}\\
        &=\frac{\text{dim}-1}{h^2}W^{\prime\prime}
    \end{aligned}
\end{equation}

\section{Governing Equations}

\subsection{Brief Explanation on Physical Values}

We divide the total fluid into a set of small mass elements. 
Considering an element $i$ has a mass $m_i$ and a density $\rho_i$ at position $\vec{r}_i$.

The scalar value $A$ at particle $i$ is $A_i$, let's consider a integral as below:

\begin{equation}
    \int \frac{A(\vec{r}^\prime)}{\rho(\vec{r}^\prime)}\rho(\vec{r}^\prime)d\vec{r}^\prime
\end{equation}

the part $\rho(\vec{r}^\prime)d\vec{r}^\prime=dm^\prime$ is the mass of the fluid element, 
thus we get:

\begin{equation}
    A_i = \sum_jm_j \frac{A_j}{\rho_j}W_{ij}
\end{equation}

Suppose $A$ is the density $\rho$, then we have:

\begin{equation}
    \rho_i = \sum_jm_j \frac{\rho_j}{\rho_j}W_{ij} = \sum_jm_j W_{ij}
\end{equation}

\subsubsection{First Derivatives}

Although first derivatives can be easily calculated by the formula above, 
for accuracy and stability, we use the following formula:

Considering a scalar $\phi$, we have:

\begin{equation}
    \begin{aligned}
        \frac{\partial (\phi A)}{\partial x_k}&=
        \phi \frac{\partial A}{\partial x_k} + A \frac{\partial \phi}{\partial x_k}\\
    \end{aligned}
\end{equation}

Thus we have:

\begin{equation}
    \frac{\partial A}{\partial x_k}=\frac{1}{\phi}
    \left(
        \frac{\partial (\phi A)}{\partial x_k} - A \frac{\partial \phi}{\partial x_k}
    \right)
\end{equation}

Use the integral above, we have:

\begin{equation}
    \begin{aligned}
        \left(
        \frac{\partial A}{\partial x_k}
    \right)_i &=
    \frac{1}{\phi_i}
        \sum_j
        \left(\frac{m_j}{\rho_j} \phi_j A_j W_{ij,k}
        - \frac{m_j}{\rho_j} \phi_j A_iW_{ij,k}
        \right)\\
        &=\frac{1}{\phi_i}
        \sum_j
        \frac{m_j}{\rho_j} \phi_j (A_j-A_i)W_{ij,k}
    \end{aligned}
\end{equation}

If $\phi=1$, we have:

\begin{equation}
    \left(
        \frac{\partial A}{\partial x_k}
    \right)_i = \sum_j \frac{m_j}{\rho_j} (A_j-A_i)W_{ij,k}
\end{equation}

If $\phi=\rho$, we have:

\begin{equation}
    \left(
        \frac{\partial A}{\partial x_k}
    \right)_i = \frac{1}{\rho_i}\sum_j m_j (A_j-A_i)W_{ij,k}
\end{equation}

this derivation is usually applied in continuity equation to attain $\nabla\cdot \vec{v}$.

\begin{equation}
    (\nabla\cdot \vec{v})_i = \frac{1}{\rho_i}\sum_j m_j (\vec{v}_j-\vec{v}_i)\cdot \nabla W_{ij}
\end{equation}

In continuity equation, we have ($\phi=1,\rho$):

\begin{equation}
    \begin{aligned}
        \frac{D\rho}{Dt} &= -\rho \nabla \cdot \vec{v}=
    \rho_i\sum_j \frac{m_j}{\rho_j} (\vec{v}_i-\vec{v}_j)\cdot \nabla W_{ij}\\
    \frac{D\rho}{Dt} &= 
    \sum_j m_j (\vec{v}_i-\vec{v}_j)\cdot \nabla W_{ij}\\
    \end{aligned}
\end{equation}

\subsubsection{Second Derivatives}

\subsection{Continuity Equation}

\begin{equation}
    \rho_i = \sum_j m_j W_{ij}
\end{equation}

or equation:

\begin{equation}
    \left(\frac{D\rho}{Dt}\right)_i = 
    \sum_j m_j (\vec{v}_i-\vec{v}_j)\cdot \nabla W_{ij}
\end{equation}

are both acceptable.

Usually, to avoid accretion of particles, we use modified continuity equation:

\begin{equation}
    \frac{D\rho}{Dt}=-\rho\nabla\cdot \vec{v}-\vec{f}
\end{equation}

in particle approximation, we have:

\begin{equation}
    \frac{D\rho}{Dt}=\sum_j m_j (\vec{v}_i-\vec{v}_j)\cdot \nabla W_{ij}-\sum_j \frac{\vec{f}_j}{\rho_j}W_{ij}
\end{equation}

\subsection{Thermal Energy Equation}

The equation for the rate of change of thermal energy per unit mass is:

\begin{equation}
    \frac{D e}{Dt}=-\left(
        \frac{p}{\rho}
    \right)\nabla\cdot \vec{v}
\end{equation}

on one hand:

\begin{equation}
    \left(\frac{De}{Dt}\right)_i=
    \left(
        \frac{p_i}{\rho_i^2}
    \right)\sum_j
    m_j(\vec{v}_i-\vec{v}_j)\cdot \nabla W_{ij}
\end{equation}

on the other hand, noting:

\begin{equation}
    \frac{De}{Dt} = 
    -\nabla\cdot\left(
        \frac{p\vec{v}}{\rho}
    \right) + \vec{v}\cdot
    \nabla\left(
        \frac{p}{\rho}
    \right)
\end{equation}

we obtain:

\begin{equation}
    \left(\frac{De}{Dt}\right)_i=
    \sum_j m_j
    \left(
        \frac{p_j}{\rho_j^2}
    \right)(\vec{v}_i-\vec{v}_j)\cdot \nabla W_{ij}
\end{equation}

taken average of the two equations above, we have:

\begin{equation}
    \left(\frac{De}{Dt}\right)_i=
    \frac{1}{2}
    \sum_j m_j
    \left(
        \frac{p_j}{\rho_j^2}+\frac{p_i}{\rho_i^2}
    \right)(\vec{v}_i-\vec{v}_j)\cdot \nabla W_{ij}
\end{equation}

\subsection{Moving Particles}

usually, we use the following equation to update the position of particles:

\begin{equation}
    \left(\frac{D\vec{r}}{Dt}\right)_i = \vec{v}_i
\end{equation}

or XSPH can be applied:

\begin{equation}
    \begin{aligned}
        \left(\frac{D\vec{r}}{Dt}\right)_i &= \vec{\hat{v}}_i
        \\
        &=\vec{v}_i + \epsilon \sum_j m_j \left(
            \frac{\vec{v}_{j}-\vec{v}_{i}}{\bar{\rho}_{ij}}W_ij
        \right)
    \end{aligned}
\end{equation}

where $\bar{\rho}_{ij}$ is the average density of particles $i$ and $j$ $(\bar{\rho}_{ij}=\frac{\rho_i+\rho_j}{2})$.

\subsection{Momentum Equation}

Let's consider this vector: $\rho\nabla p$ here:

\begin{equation}
    (\rho\nabla p)_i = \sum_j
    m_j(p_j-p_i)\nabla W_{ij}
\end{equation}

however, it's better to use the following formula:

\begin{equation}
    \frac{\nabla p}{\rho} = \nabla\left(\frac{p}{\rho}\right) + 
    \frac{p}{\rho^2}\nabla p
\end{equation}

thus we have:

\begin{equation}
    \left(
        \frac{\nabla p}{\rho}
    \right)_i = \sum_j
    m_j\left(
        \frac{p_j}{\rho_j^2}+\frac{p_i}{\rho_i^2}
    \right)\nabla W_{ij}
\end{equation}

% 这边的分部积分运用的实在有点精妙

for non-viscous fluid and non-body force's case, we have momentum equation as below:

\begin{equation}
    \left(\frac{D \vec{v}}{Dt}\right)_i = -\sum_j
    m_j\left(
        \frac{p_j}{\rho_j^2}+\frac{p_i}{\rho_i^2}
    \right)\nabla W_{ij}
\end{equation}

\subsection{Viscosity}

modify the equation above we get:

\begin{equation}
    \left(\frac{D \vec{v}}{Dt}\right)_i = -\sum_j
    m_j\left(
        \frac{p_j}{\rho_j^2}+\frac{p_i}{\rho_i^2}
        +\Pi_{ij}
    \right)\nabla W_{ij}
\end{equation}

where $\Pi_{ij}$ is the viscous force between particle $i$ and $j$:

\begin{equation}
    \Pi_{ij}=
    \begin{cases}
        \begin{aligned}
            &-\alpha \bar{c}_{ij}\mu_{ij}+\beta\mu_{ij}^2\quad (\vec{v}_i-\vec{v}_j)\cdot \vec{r}_{ij}<0\\
            &0 \quad (\vec{v}_i-\vec{v}_j)\cdot \vec{r}_{ij}\geq 0\\
        \end{aligned}
    \end{cases}
\end{equation}

where:

\begin{equation}
    \mu_{ij}=\frac{h (\vec{v}_i-\vec{v}_j)\cdot \vec{r}_{ij}}{r_{ij}^2 + \eta^2}
\end{equation}

usually, $\eta =  0.01h$ to avoid singularity.

and $\bar{c}_{ij}$ is the average sound speed of particle $i$ and $j$.

$\alpha,\beta$ are two constants to control the viscosity.

\begin{itemize}
    \item $\alpha$: produces a shear and bulk viscosity;
    \item $\beta$: used to handle high Mach number shocks.
\end{itemize}


\subsection{The Equation I use}

Let's review the basic equations:

\begin{equation}
    \begin{aligned}
        \frac{D\rho}{Dt} &= -\rho \nabla \cdot \vec{v}\\
        \frac{D\vec{v}}{Dt} &= -\frac{\nabla p}{\rho} + \vec{f} + \nu \nabla^2 \vec{v}\\
        \frac{dp}{d\rho} &= c^2
    \end{aligned}
\end{equation}

For continuity equation, we have:

\begin{equation}
    \left(
        \frac{D\rho}{Dt}
    \right)_i = \sum_j m_j (\vec{v}_i-\vec{v}_j)\cdot \nabla W_{ij}
    +\epsilon_\rho \sum_j m_j \frac{\rho_i-\rho_j}{\bar{\rho}_{ij}}W_{ij}
\end{equation}

For momentum equation, we have:

\begin{equation}
    \left(
        \frac{D\vec{v}}{Dt}
    \right)_i=
    -\sum_j m_j \left(
        \frac{p_j}{\rho_j^2}+\frac{p_i}{\rho_i^2}
    \right)\nabla W_{ij}
    +\frac{2\mu}{\rho_0^2}
    \sum_j m_j (\vec{v}_i-\vec{v}_j)\frac{|\nabla W_{ij}|}{r_{ij}}
\end{equation}

For incopressible fluid, we have $c=Const$, thus:

\begin{equation}
    p = c^2(\rho - \rho_0) + p_0
\end{equation}

\section{TODO list}

\begin{enumerate}
    \item Background grid to initialize list of particles for reducing the computational cost.
    \item Physical group to manage the behavior of different particles.
\end{enumerate}

\end{document}