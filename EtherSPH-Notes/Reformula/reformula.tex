\documentclass[10pt, oneside]{article}
\usepackage{amsmath, amsthm, amssymb, calrsfs, wasysym, verbatim, bbm, color, graphics, geometry}

\geometry{tmargin=.50in, bmargin=.50in, lmargin=.50in, rmargin = .50in}  

\usepackage{algorithm}
\usepackage[noend]{algpseudocode}

\newcommand{\R}{\mathbb{R}}
\newcommand{\C}{\mathbb{C}}
\newcommand{\Z}{\mathbb{Z}}
\newcommand{\N}{\mathbb{N}}
\newcommand{\Q}{\mathbb{Q}}
\newcommand{\Cdot}{\boldsymbol{\cdot}}

\newtheorem{theorem}{Theorem} % 定理
\newtheorem{definition}{Definition} % 定义
\newtheorem{property}{Property} % 性质
\newtheorem{convention}{Convention} % 约定
\newtheorem{remark}{Remark} % 注释
\newtheorem{lemma}{Lemma} % 引理
\newtheorem{corollary}{Corollary} % 推论


\title{SPH Method Reformula}
\author{bcynuaa}
\date{\today}

\begin{document}

\maketitle
\tableofcontents

\vspace{.25in}

\section{Kernel Function}

\subsection{Kernel Function and Its Derivative}

\subsubsection{Kernel Function'e Form}

Usually, 
kernel function has the form of:

\begin{equation}
    W(\vec{r})=W\left(
        \frac{r}{h}
    \right)=W(q)
\end{equation}

where:

\begin{equation}
    \begin{aligned}
        q &= \frac{r}{h} \quad &\Rightarrow \quad \frac{dq}{dr}= \frac{1}{h}\\
        r &= \sqrt{\sum_{k=1}^d x_k^2} \quad &\Rightarrow \quad \frac{\partial r}{\partial x_k} = \frac{x_k}{r}\\
    \end{aligned}
\end{equation}

\subsubsection{1d Derivative of Kernel Function}

\begin{equation}
    \begin{aligned}
        \frac{\partial W}{\partial x_k} &= \frac{\partial W}{\partial q}\frac{\partial q}{\partial x_k}\\
        &= \frac{\partial W}{\partial q}\frac{\partial q}{\partial r}\frac{\partial r}{\partial x_k}\\
        &= \frac{\partial W}{\partial q}\frac{1}{h}\frac{x_k}{r}\\
        &= \frac{1}{h}\frac{x_k}{r}\frac{\partial W}{\partial q}\\
        &= \frac{1}{h}\frac{x_k}{r}W^\prime
    \end{aligned}
\end{equation}

thus it's obvious that:

\begin{equation}
    \nabla W = \frac{1}{h}\frac{\vec{r}}{r}W^\prime
\end{equation}

$\nabla W$ shares the same direction with $\vec{r}$, and:

\begin{equation}
    |\nabla W| = \frac{1}{h}W^\prime
\end{equation}

\subsubsection{2d Derivative of Kernel Function}

\begin{equation}
    \begin{aligned}
        \frac{\partial^2 W}{\partial x_k^2} &= \frac{\partial}{\partial x_k}\left(
            \frac{1}{h}\frac{x_k}{r}W^\prime
        \right)\\
        &= \frac{1}{h}W^\prime \left(
            \frac{1}{r} - \frac{x_k^2}{r^3}
        \right)+
        \frac{1}{h}\frac{x_k}{r}\frac{\partial W^\prime}{\partial x_k}\\
        &= \frac{1}{h}W^\prime \frac{r^2-x_k^2}{r^3}+
        \frac{x_k}{hr} \frac{\partial W^\prime}{\partial q}\frac{\partial q}{\partial r}\frac{\partial r}{\partial x_k}\\
        &= \frac{1}{h}W^\prime \frac{r^2-x_k^2}{r^3}+
        \frac{x_k}{hr} \frac{\partial W^\prime}{\partial q}\frac{1}{h}\frac{x_k}{r}\\
        &= \frac{1}{h}W^\prime \frac{r^2-x_k^2}{r^3}+
        \frac{1}{h^2r^2}x_k^2 W^{\prime\prime}\\
        &= \frac{1}{h}W^\prime \frac{r^2-x_k^2}{r^3}+
        \frac{x_k^2}{h^2r^2} W^{\prime\prime}
    \end{aligned}
\end{equation}

thus in summary:

\begin{equation}
    \nabla^2 W = \sum_{k=1}^d \frac{\partial^2 W}{\partial x_k^2} = 
        \frac{d-1}{hr}W^\prime + \frac{1}{h^2}W^{\prime\prime}
\end{equation}

when $r\to 0$, we have:

\begin{equation}
    \nabla^2 W = \lim_{r/h\to 0} \nabla^2 W = \frac{d}{h^2}W^{\prime\prime}
\end{equation}

\subsection{Kernel Interpolation}

\subsubsection{Original Kernel Interpolation}

From the property shared with $\delta$ function:

\begin{equation}
    A(\vec{r}) = 
    \int_{\Omega} A(\vec{r}^\prime) 
    W(\vec{r} - \vec{r}^\prime) \mathrm{d}\vec{r}^\prime
\end{equation}

In particle discretization, we have:

\begin{equation}
    A_i = \sum_{j} \frac{m_j}{\rho_j} A_j W_{ij}
\end{equation}

where:

\begin{equation}
    W_{ij} = W(\vec{r}_i - \vec{r}_j)\quad \text{and} 
    \quad \sum_{j} \frac{m_j}{\rho_j} W_{ij} = 1
\end{equation}

This provides a way to interpolate the value of $A$ at any point $\vec{r}$.

\subsubsection{Gradient of Kernel Interpolation}

Consider a scalar function $A(\vec{r})$ distributed in space $\mathbb{R}^\text{dim}$. 
Let's see $\nabla A(\vec{r})$:

\begin{equation}
    \begin{aligned}
        \nabla A(\vec{r}) 
        &= \int_{\Omega} \nabla A(\vec{r}^\prime) W(\vec{r} - \vec{r}^\prime) \mathrm{d}\vec{r}^\prime \\
        &= \int_{\Omega} \nabla A(\vec{r}^\prime) W(\vec{r}^\prime - \vec{r}) \mathrm{d}\vec{r}^\prime \\
        &= \int_{\partial \Omega} A(\vec{r}^\prime) \vec{n} W(\vec{r}^\prime - \vec{r}) \mathrm{d}S^\prime
        -\int_{\Omega} A(\vec{r}^\prime) \nabla W(\vec{r}^\prime - \vec{r}) \mathrm{d}\vec{r}^\prime\\
        &= \int_{\Omega} A(\vec{r}^\prime) \nabla W(\vec{r} - \vec{r}^\prime) \mathrm{d}\vec{r}^\prime
    \end{aligned}
\end{equation}

The above formula use property of compactness and symmetry of kernel function. 
In particle discretization, we have:

\begin{equation}
    \nabla A_i = \sum_{j} \frac{m_j}{\rho_j} A_j \nabla W_{ij}
\end{equation}

where:

\begin{equation}
    \nabla W_{ij} = \nabla W(\vec{r}_i - \vec{r}_j)
\end{equation}

However, according to 'Smoothed Particle Hydrodynamic' from Monaghan 1992, 
such form of 1-st derivative interpolation is not accurate enough. 
He proposed a better way to do such interpolation. 
Supposing we have a scalar called $\phi$, let's consider $\phi\nabla A$ 's approximation:

\begin{equation}
    \begin{aligned}
        \phi\nabla A = \nabla(\phi A) - A\nabla\phi
    \end{aligned}
\end{equation}

Use interpolation above, we have:

\begin{equation}
    \begin{aligned}
        \phi_i (\nabla A)_i &= 
            \sum_j \frac{m_j}{\rho_j} \phi_j A_j \nabla W_{ij} -
            A_i\sum_j \frac{m_j}{\rho_j}  \phi_j \nabla W_{ij}\\
            &= \sum_j \frac{m_j}{\rho_j} \phi_j( A_j-A_i) \nabla W_{ij}
    \end{aligned}
\end{equation}

thus we have:

\begin{equation}
    (\nabla A)_i = \frac{1}{\phi_i}\sum_j \frac{m_j}{\rho_j} \phi_j( A_j-A_i) \nabla W_{ij} 
\end{equation}

noting that $A_{ij} = A_i - A_j$, we have:

\begin{equation}
    (\nabla A)_i = -\frac{1}{\phi_i}\sum_j \frac{m_j}{\rho_j}\phi_j A_{ij} \nabla W_{ij}
\end{equation}

For example, when we focus on $\frac{1}{\rho}\nabla p$, 
we have 2 ways to obtain its value:

\begin{equation}
    \begin{aligned}
        \phi = 1 \quad &\Rightarrow \quad 
            (\nabla p)_i = \sum_j \frac{m_j}{\rho_j} (p_j-p_i) \nabla W_{ij} \\
        \phi_i = \frac{1}{\rho} \quad &\Rightarrow \quad 
            \frac{1}{\rho}\nabla p = \nabla \left(
                \frac{p}{\rho}
            \right)+
            \frac{p}{\rho^2}\nabla \rho\\
            &\Rightarrow\quad
            \left(\frac{\nabla p}{\rho}\right)_i=
            \sum_j m_j \left(
                \frac{p_j}{\rho_j^2}+\frac{p_i}{\rho_i^2}
            \right) \nabla W_{ij}
    \end{aligned}
\end{equation}

These 2 ways are the most widely used in SPH method.

\subsubsection{Divergence of Kernel Interpolation}

Now let's consider a vector $\vec{A}$ distributed in space $\mathbb{R}^\text{dim}$.
Let's see $\nabla\cdot\vec{A}$:

\begin{equation}
    \begin{aligned}
        \nabla\cdot\vec{A}(\vec{r}) &= \int_{\Omega} \nabla\cdot\vec{A}(\vec{r}^\prime) W(\vec{r} - \vec{r}^\prime) \mathrm{d}\vec{r}^\prime \\
        &= \int_{\Omega} \nabla\cdot\vec{A}(\vec{r}^\prime) W(\vec{r}^\prime - \vec{r}) \mathrm{d}\vec{r}^\prime \\
        &= \int_{\partial \Omega} \vec{A}(\vec{r}^\prime) \cdot \vec{n} W(\vec{r}^\prime - \vec{r}) \mathrm{d}S^\prime -
        \int_{\Omega} \vec{A}(\vec{r}^\prime) \cdot \nabla W(\vec{r}^\prime - \vec{r}) \mathrm{d}\vec{r}^\prime\\
        &= \int_{\Omega} \vec{A}(\vec{r}^\prime) \cdot \nabla W(\vec{r} - \vec{r}^\prime) \mathrm{d}\vec{r}^\prime
    \end{aligned}
\end{equation}

In particle discretization, we have:

\begin{equation}
    (\nabla\cdot\vec{A})_i = \sum_{j} \frac{m_j}{\rho_j} \vec{A}_j \cdot \nabla W_{ij}
\end{equation}

where:

\begin{equation}
    \nabla W_{ij} = \nabla W(\vec{r}_i - \vec{r}_j)
\end{equation}

However, this way o interpolation is not accurate enough. 
Let's consider a scalar $\phi$ here:

\begin{equation}
    \phi \nabla\cdot \vec{A} = \nabla\cdot(\phi \vec{A}) - \vec{A}\cdot\nabla\phi
\end{equation}

Use interpolation above, we have:

\begin{equation}
    \begin{aligned}
        \phi_i (\nabla\cdot \vec{A})_i &= 
            \sum_j \frac{m_j}{\rho_j} \phi_j \vec{A}_j \cdot \nabla W_{ij} -
            \vec{A}_i\cdot\sum_j \frac{m_j}{\rho_j}  \phi_j \nabla W_{ij}\\
            &=
            \sum_j \frac{m_j}{\rho_j} \phi_j (\vec{A}_j-\vec{A}_i) \cdot \nabla W_{ij}\\
            &=
            -\sum_j \frac{m_j}{\rho_j} \phi_j \vec{A}_{ij} \cdot \nabla W_{ij}
    \end{aligned}
\end{equation}

where $\vec{A}_{ij} = \vec{A}_i - \vec{A}_j$. 
For example, when we calculate $\rho \nabla\cdot\vec{v}$, interpolation above reveals:

\begin{equation}
    (\rho \nabla\cdot \vec{v})_i = -\sum_j m_j \vec{v}_{ij} \cdot \nabla W_{ij}
\end{equation}

such interpolation is widely used in SPH method.

\subsubsection{Laplacian of Kernel Interpolation}

Now let's consider a scalar function $A(\vec{r})$ distributed in space $\mathbb{R}^\text{dim}$.

\begin{equation}
    \begin{aligned}
        \nabla^2 A(\vec{r}) &= \int_{\Omega} \nabla^2 A(\vec{r}^\prime) W(\vec{r} - \vec{r}^\prime) \mathrm{d}\vec{r}^\prime \\
        &= \int_{\Omega} \nabla^2 A(\vec{r}^\prime) W(\vec{r}^\prime - \vec{r}) \mathrm{d}\vec{r}^\prime \\
        &= \int_{\partial \Omega} \nabla A(\vec{r}^\prime) \cdot \vec{n} W(\vec{r}^\prime - \vec{r}) \mathrm{d}S^\prime -
        \int_{\Omega} \nabla A(\vec{r}^\prime) \cdot \nabla W(\vec{r}^\prime - \vec{r}) \mathrm{d}\vec{r}^\prime\\
        &= -\int_{\Omega} \nabla A(\vec{r}^\prime) \cdot \nabla W(\vec{r}^\prime - \vec{r}) \mathrm{d}\vec{r}^\prime\\
        &= -\int_{\partial \Omega} A(\vec{r}^\prime)\vec{n} \cdot \nabla W(\vec{r}^\prime - \vec{r}) \mathrm{d}S^\prime +
        \int_{\Omega} A(\vec{r}^\prime) \nabla^2 W(\vec{r}^\prime - \vec{r}) \mathrm{d}\vec{r}^\prime\\
        &= \int_{\Omega} A(\vec{r}^\prime) \nabla^2 W(\vec{r} - \vec{r}^\prime) \mathrm{d}\vec{r}^\prime
    \end{aligned}
\end{equation}

In particle discretization, we have:

\begin{equation}
    (\nabla^2 A)_i = \sum_{j} \frac{m_j}{\rho_j} A_j \nabla^2 W_{ij}
\end{equation}

The same problem with this formula is accuracy. 
For this part, 
I consult a large number of references, 
but each reference has its own way of interpolation.

in 1985, Brookshaw proposed a way based on Taylor expansion:

\begin{equation}
    (\nabla^2 A)_i = -\sum_{j} \frac{m_j}{\rho_j} A_{ij}\frac{2|\nabla W_{ij}|}{|r_{ij}|}
\end{equation}

However, after I reading the original paper, I find he only proposed a 1-D case's formula, 
In Dan Koschier's paper, he directly use the above formula in 3-D case. 
And in another paper from Price, 2012, he use the following formula:

\begin{equation}
    (\nabla^2 A)_i = 2\sum_{j} \frac{m_j}{\rho_j} A_{ij}\frac{|\nabla W_{ij}|}{|r_{ij}|}
\end{equation}

My math is not good enough to prove the correctness of this formula. 
But from my personal formula, I think the following formula is more reasonable:

\begin{equation}
    \begin{aligned}
        \nabla^2 A(\vec{r})&= 
        \int_{\Omega} \nabla A(\vec{r}^\prime) \cdot \nabla W(\vec{r}- \vec{r}^\prime ) \mathrm{d}\vec{r}^\prime\\
        &\approx
        \sum_j \frac{m_j}{\rho_j} \frac{A_i-A_j}{|\vec{r}_i-\vec{r}_j|^2}
        (\vec{r}_i - \vec{r}_j)\cdot \nabla W_{ij}\\
        &=\sum_j \frac{m_j}{\rho_j}\frac{A_{ij}}{r_{ij}^2}\vec{r}_{ij}\cdot \nabla W_{ij}
    \end{aligned}
\end{equation}

From the above formula, $|\nabla W_{ij}|$ shares the same direction with $\vec{r}_{ij}$,
thus we have:

\begin{equation}
    \nabla^2 A(\vec{r})\approx
    \sum_j \frac{m_j}{\rho_j}\frac{A_{ij}}{r_{ij}}|\nabla W_{ij}|
\end{equation}

this result is quite similar to Price, 2012's formula except for the coefficient '$2$'.
Currently, I have no idea on why there is a '$2$' in his formula. 
A possible explanation is that it comes from the Taylor expansion of $W$.

\subsubsection{Vector Second Derivative of Kernel Interpolation}

I copy this from Price, 2012's paper but really have no idea on how to prove it:

\begin{equation}
    \begin{aligned}
        (\nabla^2 \vec{A})_i&=-2\sum_j \frac{m_j}{\rho_j}\vec{A}_{ij}\frac{|\nabla W_{ij}|}{r_{ij}}\\
    \end{aligned}
\end{equation}

What's more, for a divergence-free vector field $\vec{A}$ (which has $\nabla\cdot\vec{A}=0$), 
Price proposed a way to calculate $\nabla^2 \vec{A}$:

\begin{equation}
    (\nabla^2 \vec{A})_i=2(d+2)\sum_{j} 
    \frac{m_j}{\rho _j} \frac{\vec{A}_{ij}\cdot \vec{r}_{ij}}{r_{ij}^2}\nabla W_{ij}
\end{equation}

\section{Governing Equation And Discretization}

\subsection{Weak Compressible SPH And Governing Equation}

This method allows density to vary in space, instead of keeping it constant. 
This method is widely used in SPH method.

The governing equation is as follows (Lagrange form):

\begin{equation}
    \begin{aligned}
        \frac{\mathrm{d} \rho}{\mathrm{d} t} &= -\rho \nabla\cdot \vec{v}\\
        \frac{\mathrm{d} \vec{v}}{\mathrm{d} t} &= -\frac{1}{\rho}\nabla p + \vec{g} + \nu \nabla^2 \vec{v}\\
        \frac{\mathrm{d} \vec{r}}{\mathrm{d} t} &= \vec{v}\\
        \frac{\mathrm{d}p}{\mathrm{d} \rho} &= c^2
    \end{aligned}
\end{equation}

WCSPH allows density to vary in space, thus pressure without $p_0$ can be written as:

\begin{equation}
    p = c^2(\rho - \rho_0)
\end{equation}

Here for initializing the whole fluid system, 
we need to calculate the initial pressure $p$ and density $\rho$.
Usually, pressure obeys the stable hydrostatic equation:

\begin{equation}
    p = \rho g h+ p_0
\end{equation}

In calculation, we can set $p_0=0$ for convenience. 
And density can be calculated from the following equation:

\begin{equation}
    \rho = \rho_0 + \frac{p}{c^2}
\end{equation}

For $c$ in water is $1481m/s$, $\Delta\rho$ would be small.

\subsection{A Simple Solver for WCSPH}

Thus a whole fluid solver is proposed as follows:

\begin{algorithm}[H]
    \caption{WCSPH Simple Solver}
    \begin{algorithmic}[1]
        \State \textbf{Input:} $\rho_0, c, \nu, \vec{g}, \Delta t$ etc.
        \State \textbf{Output:} $\rho, \vec{v}, p$
        \State \textbf{Initialize:}:
        \begin{itemize}
            \item $p = \rho_0 g h$
            \item $\rho = \rho_0 + \frac{p}{c^2}$
            \item $\vec{v} = \vec{v}_0$
        \end{itemize}
        \While{current time step < total time step}
            \For{each particle $i$}
                \State Update $\rho$ using $\frac{d\rho}{dt}= -\rho \nabla\cdot \vec{v}$
            \EndFor
            \For{each particle $i$}
                \State Update $\vec{v}$ to $\vec{v}^*$ without pressure using $\frac{d\vec{v}}{dt}= \vec{g} + \nu \nabla^2 \vec{v}$
            \EndFor
            \For{each particle $i$}
                \State Update $p$ using $p = c^2(\rho - \rho_0)$
            \EndFor
            \For{each particle $i$}
                \State Update $\vec{v}$ to $\vec{v}^{n+1}$ with pressure using $\frac{d\vec{v}}{dt}= -\frac{1}{\rho}\nabla p$
                \State Update $\vec{r}$ using $\frac{d\vec{r}}{dt}= \vec{v}$
            \EndFor
        \EndWhile
    \end{algorithmic}
\end{algorithm}

\subsection{Discretization of Governing Equation}

\subsubsection{Continuity Discretization}

\begin{equation}
    \left(\frac{d\rho}{dt}\right)_i = 
    \sum_j m_j (\vec{v}_i-\vec{v}_j)\cdot \nabla W_{ij}
\end{equation}

here we obtain the density of each particle at the next time step:

\begin{equation}
    \rho_i^{n+1} = \rho_i^n + \Delta t \sum_j m_j (\vec{v}_i-\vec{v}_j)\cdot \nabla W_{ij}
\end{equation}

However, this step ususally needs correction (accoring to SmoothedParticles.jl):

\begin{equation}
    \left(\frac{d\rho}{dt}\right)_i = 
    \sum_j m_j (\vec{v}_i-\vec{v}_j)\cdot \nabla W_{ij}+
    2\epsilon_\rho\sum_j m_j (\rho_i-\rho_j)\frac{|\nabla W_{ij}|}{r_{ij}}
\end{equation}

here $\epsilon_\rho$ is $1\times 10^{-6}$.

note: I think here can be written as a sparse matrix form to ensure divergence-free.

Wait to be done in future work.

\subsubsection{Motion Equation Discretization Without Pressure}

\begin{equation}
    \frac{d\vec{v}}{dt} = \vec{g} + \frac{\mu}{\rho} \nabla^2 \vec{v}
\end{equation}

Here we use the following formula to calculate $\nabla^2 \vec{v}$:

\begin{equation}
    \nabla^2 \vec{v} = 2(d+2)\sum_j
    \frac{m_j}{\rho_j}\frac{\vec{v}_{ij}\cdot \vec{r}_{ij}}{r_{ij}^2}\nabla W_{ij}
\end{equation}

To avoid singularity, we modify the above formula as follows:

\begin{equation}
    \nabla^2 \vec{v} = 2(d+2)\sum_j
    \frac{m_j}{\rho_j}\frac{\vec{v}_{ij}\cdot \vec{r}_{ij}}{r_{ij}^2+0.01h^2}\nabla W_{ij}
\end{equation}

Actucally, after reading Price's formula and Koschier's reference and Delong Xu's 
form, I'm confused this part has no unified form. I simply apply current formula

\begin{equation}
    \vec{f}_{\text{vis}}=
    \mu \sum_{j} m_j \frac{2(d+2)}{\rho_i\rho_j}
    \frac{\vec{v}_{ij}\cdot \vec{r}_{ij}}{r_{ij}^2+0.01h^2}\nabla W_{ij}
\end{equation}

\subsubsection{Pressure Discretization}

\begin{equation}
    p_i^{n+1} = c^2 \left(
        \rho_i^{n+1} - \rho_0
    \right)
\end{equation}

\subsubsection{Motion Equation Discretization With Pressure}

\begin{equation}
    \left(\frac{d\vec{v}}{dt}\right)_i = -
    \sum_j m_j \left(
        \frac{p_i^{n+1}}{\rho_i^{n+1}\rho_i^{n+1}}+\frac{p_j^{n+1}}{\rho_j^{n+1}\rho_j^{n+1}}
    \right) \nabla W_{ij}
\end{equation}

Thus we have:

\begin{equation}
    \vec{v}_i^{n+1} = \vec{v}_i^* + \Delta t 
        \sum_j m_j \left(
        \frac{p_i^{n+1}}{\rho_i^{n+1}\rho_i^{n+1}}+\frac{p_j^{n+1}}{\rho_j^{n+1}\rho_j^{n+1}}
    \right) \nabla W_{ij}
\end{equation}

Monaghan proposed an artificial viscosity term to avoid particle penetration:

\subsubsection{Position Discretization}

\begin{equation}
    \vec{r}_i^{n+1} = \vec{r}_i^n + \Delta t \vec{v}_i^{n+1}
\end{equation}

Actually, here can also be correted by some unreasonable addition such like XSPH method 
proposed by Monaghan, 
but I'm not going to use it for current SPH code.

\subsection{The Progress of This My SPH Code}

In summary, 
I plan to design the SPH code as following procedure:

\begin{algorithm}[H]
    \caption{WCSPH Simple Solver}
    \begin{algorithmic}[1]
        \State \textbf{Input:} $\rho_0, c, \nu, \vec{g}, \Delta t$ etc.
        \State \textbf{Output:} $\rho, \vec{v}, p$
        \State \textbf{Initialize:}: $\rho, \vec{v}, p$
        \While{current time step < total time step}
            \For{each particle $i$}
                \State Update $\rho$ using $\frac{d\rho}{dt}= -\rho \nabla\cdot \vec{v}$
            \EndFor
            \For{each particle $i$}
                \State Update $p$ using $p = c^2(\rho - \rho_0)+p_0$
            \EndFor
            \For{each particle $i$}
                \State Update $\vec{v}$ to $\vec{v}^{n+1}$ with pressure using $\frac{d\vec{v}}{dt}= -\frac{1}{\rho}\nabla p+\frac{\mu}{\rho} \nabla^2 \vec{v}+\vec{g}$
                \State Update $\vec{r}$ using $\frac{d\vec{r}}{dt}= \vec{v}$
            \EndFor
        \EndWhile
    \end{algorithmic}
\end{algorithm}

However, this procedure is not the same as any other SPH code. 
The time discretization quite relies on the SPH interpolation form. 
Different interpolation form will lead to different time discretization, 
and the code procedure will be different. 
My current code is based on the above procedure, 
but I'm not sure whether it's correct or not.

\subsection{Boundary Handling}

According to Koschier's paper, he proposed a simple boundary handling method.

The boundary particle can be seen as a mirror particle of the real particle, 
which produces the same force as the real particle but in the opposite direction, 
especially the pressure force.

\begin{equation}
    \left(-\frac{1}{\rho}\nabla p\right)_{i}=
    -\frac{2m_{i}p_i}{\rho_{i}^2}\sum_{j,b}\nabla W_{ij,b}
\end{equation}

However, this only works for $\rho_i>\rho_0$ and $p_i>0$,  
otherwise, this should be written as:

\begin{equation}
    \left(-\frac{1}{\rho}\nabla p\right)_{i}=
    -m_ip_i\sum_{j,b}\left(
        \frac{1}{\rho_i^2}+\frac{1}{\rho_0^2}
    \right)
    \nabla W_{ij,b}
\end{equation}

\end{document}