\section{SPH 方法回顾}

\begin{frame}
    一个物理场量 $A(\vec{r})$ 可以用其在空间中分布的加权积分得到:
    \begin{equation}
        A(\vec{r}) = \int A(\vec{r}^\prime) W(\vec{r} - \vec{r}^\prime, h) \mathrm{d} \vec{r}^\prime
    \end{equation}
    其中 $W(\vec{r} - \vec{r}^\prime, h)$ 是光滑核函数,$h$ 是光滑长度。
    这是一个与 $\delta(\vec{r})$ 相似的函数,但是具有有限的支撑域。
    一个物理量的核函数插值,仅由其在支撑域内的物理量决定。
    同时,该物理量梯度经分部积分和边界舍入,也可以表达为:
    \begin{equation}
        \nabla A(\vec{r}) = \int A(\vec{r}^\prime) \nabla W(\vec{r} - \vec{r}^\prime, h) \mathrm{d} \vec{r}^\prime
    \end{equation}
    在物理量被一颗颗粒子携带的时候,
    上述积分可以表示为对支撑域内所有粒子的加权求和:
    \begin{equation}
        \begin{aligned}
            A_i &= \sum_j \frac{m_j}{\rho_j} A_j W_{ij} \\
            \nabla A_i &= \sum_j \frac{m_j}{\rho_j} A_j \nabla W_{ij}
            \quad \nabla W_{ij} \parallel \vec{r}_{ij}
        \end{aligned}
    \end{equation}
\end{frame}

\begin{frame}
    虽然据此已经可以离散出含导数的物理方程,
    但精度并不高,
    因此多种离散形式被提出,其核心思想大致如下,
    对 $\nabla$ 算子外用指标 $i$ ,算子内用指标 $j$:
    \begin{equation}
        \begin{aligned}
            \phi \nabla A &= \nabla (\phi A) - A \nabla \phi\\
            &= \sum_j \frac{m_j}{\rho_j} \phi_j A_j \nabla W_{ij}
            - \sum_j \frac{m_j}{\rho_j} A_i \phi_j \nabla W_{ij}\\
            &= \sum_j \frac{m_j}{\rho_j} \phi_j (A_j - A_i) \nabla W_{ij}
        \end{aligned}
    \end{equation}
    比如对于项 $\rho\nabla\cdot \vec{v}$ 的离散形式如($\phi=\rho$):
    \begin{equation}
        \rho_i \nabla \cdot \vec{v}_i = -\sum_j m_j \vec{v}_{ij} \cdot \nabla W_{ij}
    \end{equation}
    而对于 $\frac{1}{\rho}\nabla p = \nabla \left(\frac{p}{\rho}\right) + \frac{p}{\rho^2}\nabla \rho$,
    则有形式如($\phi=\frac{1}{\rho}$):
    \begin{equation}
        \frac{1}{\rho_i}\nabla p_i = \sum_j m_j \left(\frac{p_i}{\rho_i^2} + \frac{p_j}{\rho_j^2}\right) \nabla W_{ij}
    \end{equation}
\end{frame}

\begin{frame}
    在 Lagrange 描述下,不可压流体的基本方程为,
    暂时不考虑能量方程。
    \begin{equation}
        \begin{aligned}
            \frac{\mathrm{d} \rho}{\mathrm{d} t} &= -\rho \nabla \cdot \vec{v} \\
            \frac{\mathrm{d} \vec{v}}{\mathrm{d} t} &= -\frac{1}{\rho} \nabla p + \vec{g} + \nu \nabla^2 \vec{v}\\
            \frac{\mathrm{d} \vec{x}}{\mathrm{d} t} &= \vec{v}
        \end{aligned}
    \end{equation}
    目前 SPH 方法广为使用的方式是弱可压 WCSPH 方法,
    其允许流体的密度在一定范围内变化。
    并且给出了状态方程如下:
    \begin{equation}
        p = \frac{c_0^2\rho_0}{\gamma}
        \left[\left(\frac{\rho}{\rho_0}\right)^\gamma - 1\right]=
        \frac{c_0^2\rho_0}{\gamma}
        \left[\left(1+\frac{\rho-\rho_0}{\rho_0}\right)^\gamma - 1\right]
    \end{equation}
    在 $\rho$ 变化不大时,方程可以退化为 $p=c_0^2(\rho-\rho_0)$。
\end{frame}

\begin{frame}
    根据光滑核函数插值,
    我们可以将 SPH 的控制方程离散成光滑核函数的插值形式:
    \begin{equation}
        \begin{aligned}
            \frac{\mathrm{d} \rho_i}{\mathrm{d} t} &= \rho_i \sum_j m_j \vec{v}_{ij} \cdot \nabla W_{ij} \\
            \frac{\mathrm{d} \vec{v}_i}{\mathrm{d} t} &= -\sum_j m_j \left(\frac{p_i}{\rho_i^2} + \frac{p_j}{\rho_j^2}\right) \nabla W_{ij} + \vec{g} + \nu \nabla^2 \vec{v}_i
        \end{aligned}
    \end{equation}
    其中粘性项的离散形式比较多样,
    各种参考文献中给出的结果不尽相同:
    \begin{equation}
        \nu \nabla^2 \vec{v}_i = 
        \sum_j m_j
        \frac{4 (\mu_i + \mu_j) \vec{r}_{ij}\cdot \nabla W_{ij}}{(\rho_i + \rho_j)^2 (r_{ij}^2 + 0.01h^2)}\vec{v}_{ij}
    \end{equation}
\end{frame}

\begin{frame}
    一种可能的推导过程如下,将某速度矢量作泰勒展开:
    \begin{equation}
        \vec{v}_i \approx
        \vec{v}_j + (\vec{r}_i - \vec{r}_j)\cdot \nabla\vec{v}_j
        \to
        \vec{v}_{ij} \approx \vec{r}_{ij}\cdot \nabla\vec{v}_j
    \end{equation}
    其中对二维情况,
    $\nabla \vec{v}_j$ 是一个 $2\times 2$ 的矩阵,
    \begin{equation}
        \nabla\vec{v}_j =
    \begin{bmatrix}
        \frac{\partial v_{j,x}}{\partial x} & \frac{\partial v_{j,y}}{\partial x} \\
        \frac{\partial v_{j,x}}{\partial y} & \frac{\partial v_{j,y}}{\partial y} 
    \end{bmatrix}
    \end{equation}
    进而导出:
        \begin{equation}
            \begin{aligned}
                \nabla^2\vec{v}_i &= [\nabla\cdot (\nabla\vec{v})]_i\\
                &\approx \sum_j \frac{m_j}{\rho_j} \nabla W_{ij}
                \cdot \nabla\vec{v}_j\\
            \end{aligned}
        \end{equation}
\end{frame}

\begin{frame}
    对于$\nabla W_{ij} = \frac{\vec{r}_{ij}}{r_{ij}}\frac{1}{h}W^\prime_{ij}$:
    \begin{equation}
        \vec{r}_{ij}\cdot W_{ij} = \frac{r_{ij}}{h}W^\prime_{ij}
    \end{equation}
    \begin{equation}
        \begin{aligned}
            \nabla^2\vec{v}_i &= 
        \sum_j \frac{m_j}{\rho_j} \frac{1}{h r_{ij}}W^\prime_{ij}
        (\vec{r}_{ij} \cdot \nabla\vec{v}_j)\\
        &=
        \sum_j \frac{m_j}{\rho_j} \frac{1}{h r_{ij}}W^\prime_{ij}\vec{v}_{ij}\\
        &=
        \sum_j \frac{m_j}{\rho_j} \frac{1}{r_{ij}^2} \frac{r_{ij}}{h}W^\prime_{ij}\vec{v}_{ij}\\
        &=
        \sum_j \frac{m_j}{\rho_j} \frac{\vec{r}_{ij}\cdot \nabla W_{ij}}{r_{ij}^2} \vec{v}_{ij}\\
        \end{aligned}
    \end{equation}
    为了避免奇异,在分母上添加项,我的推导和 SPhysics 中的形式一致:
    \begin{equation}
        \nabla^2\vec{v}_i = 
        \sum_j \frac{m_j}{\rho_j} \frac{\vec{r}_{ij}\cdot \nabla W_{ij}}{r_{ij}^2+0.01h^2} \vec{v}_{ij}
    \end{equation}
\end{frame}

\begin{frame}
    上述推导是我能够理解的也认为是正确的,
    但对于粘性项似乎文献中给出了千奇百怪的形式,
    有的甚至干脆直接差一个系数,比如 Jan Bender, 2020 给出的形式为:
    \begin{equation}
        \nabla^2\vec{v}_i =
        2(d+2)\sum_j
        \frac{m_j}{\rho_j}
        \frac{\vec{v}_{ij}\cdot \vec{r}_{ij}}{r_{ij}^2 + 0.01h^2}
        \nabla W_{ij}
    \end{equation}
    Colagrossi 和 Aristodemo 等给出的公式则为:
    \begin{equation}
        \nu_0 \nabla^2\vec{v}_i =
        \sum_j 
        \frac{4(d+2)\nu_0 \vec{v}_{ij}\cdot \vec{r}_{ij}}{(\rho_i + \rho_j)(r_{ij}^2 + 0.01h^2)}
        \nabla W_{ij}
    \end{equation}
    Price, 2012 则给出的公式为:
    \begin{equation}
        \nabla^2\vec{v}_i =
        -2\sum_j
        \frac{m_j}{\rho_j}
        \vec{v}_{ij}\frac{\vec{r}_{ij}\cdot\nabla W_{ij}}{r_{ij}^2+0.01h^2}
    \end{equation}
    这些推导都没有详尽的解释,看起来有点关系,
    但其实表达的含义完全不同。
    这也为后续的推导造成了比较大的影响,尤其是壁面摩擦。
\end{frame}

\begin{frame}
    \begin{algorithm}[H]
        \caption{WCSPH Simple Solver}
        \begin{algorithmic}[1]
            \State \textbf{Input:} $\rho_0, c, \nu, \vec{g}, \Delta t$ etc.
            \State \textbf{Output:} $\rho, \vec{v}, p$
            \While{current time step < total time step}
                \State neighbour = findNeighbour(particles, radius)
                \For{each neighbour $i,j$}
                    \State Update $\rho$ using $\frac{d\rho}{dt}= -\rho \nabla\cdot \vec{v}$
                \EndFor
                \For{each particle $i$}
                    \State Update $p$ using $p = c^2(\rho - \rho_0)+p_0$
                \EndFor
                \For{each neighbour $i,j$ (wall includes compulsive force)}
                    \State $\vec{f}_{\text{p}}= -\sum_j m_j \left(\frac{p_i}{\rho_i^2} + \frac{p_j}{\rho_j^2}\right) \nabla W_{ij}$
                    \State $\vec{f}_{\text{v}}= \sum_j m_j \frac{4 (\mu_i + \mu_j) \vec{r}_{ij}\cdot \nabla W_{ij}}{(\rho_i + \rho_j)^2 (r_{ij}^2 + 0.01h^2)}\vec{v}_{ij}$
                \EndFor
                \For{each particle $i$}
                    \State Update $\vec{v}$ to $\vec{v}^{n+1}$ with pressure using $\frac{d\vec{v}}{dt}= -\frac{1}{\rho}\nabla p+\frac{\mu}{\rho} \nabla^2 \vec{v}+\vec{g}$
                    \State Update $\vec{r}$ using $\frac{d\vec{r}}{dt}= \vec{v}$
                \EndFor
            \EndWhile
        \end{algorithmic}
    \end{algorithm}
\end{frame}