\section{其他技术的讨论}

\subsection{XSPH 方法}

\begin{frame}
    在之前的程序中,
    XSPH 方法被采用:
    \begin{equation}
        \frac{\mathrm{d}\vec{r}_i}{\mathrm{d}t}=
        \vec{v}_i-
        \epsilon\sum_j
        \frac{m_j}{\rho_j}
        \vec{v}_{ij}W_{ij}
    \end{equation}
    但该方法仅限于层流求解,
    并不适用于开放水体的求解。
    比如流体回注入水槽的时候,
    若采用 XSPH 会导致平稳水面被注入流体破坏,
    甚至很容易产生异常负压区,进而发生数值不稳定。


    但是对于泊肃叶流和顶盖驱动流类似算例,
    采用 XSPH 方法确实可以得到较好的结果。
\end{frame}

\subsection{$\delta^+$ SPH 密度修正器}

\begin{frame}

$\delta^+$ SPH 密度修正器可以有效地稳定密度场,进而稳定压力场。

\begin{equation}
    \left(
        \frac{\partial \rho}{\partial t}
    \right)_i
    =
    \sum_j m_j \vec{u}_{ij} \cdot \nabla W_{ij}
    +
    h c_0 \delta
    \sum_j  \frac{m_j}{\rho_j} \vec{D}_{ij}\cdot \nabla W_{ij}
\end{equation}
$\vec{D}_{ij}$ 计算如下。 首先提出一个核修正矩函数:
$[L]_i$ :
\begin{equation}
    [L]_i = \left[\sum_j \frac{m_j}{\rho_j}\vec{r}_{ij}\nabla W_{ij}\right]^{-1}
\end{equation}
$[L]_i$ 是一个 $d\times d$ 的矩阵,$d$ 为维度。
粒子致密时等价于单位阵。
进而提出修正的密度梯度 $\nabla \rho_i^L$ :
\begin{equation}
    \nabla \rho_i^L = \sum_j \frac{m_j}{\rho_j}\rho_{ij}
    [L]_i \cdot \nabla W_{ij}
\end{equation}
最后由 $\vec{D}_{ij}$ 给出:
\begin{equation}
    \vec{D}_{ij} = 
    \left[
        2\rho_{ij} - 
        \left(
            \nabla \rho_i^L + \nabla \rho_j^L
        \right)\cdot \vec{r}_{ij}
    \right]
    \frac{\vec{r}_{ij}}{r_{ij}^2 + 0.01h^2}
\end{equation}
\end{frame}

\begin{frame}
    $\delta$ 是一个无量纲参数,通常被设定为 $0.1$.
Monaghan 提出了一个修正器,可以有效地稳定密度场,进而稳定压力场。

$\delta^+$ SPH 模型有时甚至可以在弱可压(WCSPH)情形下给出比不可压(ICSPH)更好的结果。


我也在程序中实现了该修正器,
但不幸的是这个修正器对于计算量的增加是非常大的,
大约二位情形下会有 $6$ 倍的计算消耗提升,
这也可能是 SPH 方法不利用速度一阶导计算速度二阶导的原因(会涉及到两次邻域插值过程,对计算消耗很大)。

\begin{equation}
    \rho_i = \frac{\sum_j m_j W_{ij}}{\sum_j \frac{m_j}{\rho_j}W_{ij}}
\end{equation}

相对而言,频繁地使用上述核函数修正来稳定密度场可能从效果来看更好,
在实践中
$\delta^+$ 似乎学术意义大于其实际价值。
\end{frame}

\subsection{未来的工作安排}

\begin{frame}
    预计未来的工作计划:
    \begin{itemize}
        \item 打算上并行,在思考 CUDA 或者 MPI 的改版程序,现在的计算效率问题很大;
        \item 对于不同问题考虑不同的壁面摩擦,而不是用统一的壁面粘性模型,甚至可以自己提出一种壁面模型;
        \item 考虑不同的问题,比如传热等,物料或者多相流问题等。
        \item ...
    \end{itemize}
    *目前个人最想做的是改一个更强的并行版,现在的线程并行虽然有一定的效率,
    但遇到三维的大规模粒子计算还是捉襟见肘。
    
    *另外我觉得 Cruchaga 给的实验结果是非常好的算例比较,
    只要提出的壁面模型能和其实验值吻合,
    那么就能说明摩擦模型的可行性,
    进而用于更多的问题。
\end{frame}