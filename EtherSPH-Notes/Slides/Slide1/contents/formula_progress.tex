\section{公式查询及推导进度}

\subsection{核函数插值的公式推导}

\begin{frame}
    \frametitle{插值核函数的一般形式}
Usually, kernel function has the form of:

\begin{equation}
    W(\vec{r})=W\left(
        \frac{r}{h}
    \right)=W(q)
\end{equation}

where:

\begin{equation}
    \begin{aligned}
        q &= \frac{r}{h} \quad &\Rightarrow \quad \frac{dq}{dr}= \frac{1}{h}\\
        r &= \sqrt{\sum_{k=1}^d x_k^2} \quad &\Rightarrow \quad \frac{\partial r}{\partial x_k} = \frac{x_k}{r}\\
    \end{aligned}
\end{equation}
\end{frame}


\begin{frame}
    \frametitle{插值核函数的梯度}
    \begin{equation}
        \begin{aligned}
            \frac{\partial W}{\partial x_k} &= \frac{\partial W}{\partial q}\frac{\partial q}{\partial x_k}\\
            &= \frac{\partial W}{\partial q}\frac{\partial q}{\partial r}\frac{\partial r}{\partial x_k}\\
            &= \frac{\partial W}{\partial q}\frac{1}{h}\frac{x_k}{r}\\
            &= \frac{1}{h}\frac{x_k}{r}\frac{\partial W}{\partial q}\\
            &= \frac{1}{h}\frac{x_k}{r}W^\prime
        \end{aligned}
    \end{equation}
    thus it's obvious that:
    \begin{equation}
        \nabla W = \frac{1}{h}\frac{\vec{r}}{r}W^\prime
        ,\quad |\nabla W| = \frac{1}{h}|W^\prime|
    \end{equation}
\end{frame}


\begin{frame}
    \frametitle{插值核函数的拉普拉斯算子值}
    
\begin{equation}
    \begin{aligned}
        \frac{\partial^2 W}{\partial x_k^2} &= \frac{\partial}{\partial x_k}\left(
            \frac{1}{h}\frac{x_k}{r}W^\prime
        \right)\\
        &= \frac{1}{h}W^\prime \left(
            \frac{1}{r} - \frac{x_k^2}{r^3}
        \right)+
        \frac{1}{h}\frac{x_k}{r}\frac{\partial W^\prime}{\partial x_k}\\
        &= \frac{1}{h}W^\prime \frac{r^2-x_k^2}{r^3}+
        \frac{x_k}{hr} \frac{\partial W^\prime}{\partial q}\frac{\partial q}{\partial r}\frac{\partial r}{\partial x_k}\\
        &= \frac{1}{h}W^\prime \frac{r^2-x_k^2}{r^3}+
        \frac{x_k}{hr} \frac{\partial W^\prime}{\partial q}\frac{1}{h}\frac{x_k}{r}\\
        &= \frac{1}{h}W^\prime \frac{r^2-x_k^2}{r^3}+
        \frac{1}{h^2r^2}x_k^2 W^{\prime\prime}\\
        &= \frac{1}{h}W^\prime \frac{r^2-x_k^2}{r^3}+
        \frac{x_k^2}{h^2r^2} W^{\prime\prime}
    \end{aligned}
\end{equation}
\end{frame}


\begin{frame}
    \frametitle{插值核函数的拉普拉斯算子值}
    \begin{equation}
        \nabla^2 W = \sum_{k=1}^d \frac{\partial^2 W}{\partial x_k^2} = 
            \frac{d-1}{hr}W^\prime + \frac{1}{h^2}W^{\prime\prime}
    \end{equation}
    
    when $r\to 0$, we have:
    
    \begin{equation}
        \nabla^2 W = \lim_{r/h\to 0} \nabla^2 W = \frac{d}{h^2}W^{\prime\prime}
    \end{equation}
\end{frame}

\subsection{核函数插值格式}

\begin{frame}
    \frametitle{梯度的核函数插值格式}
    From the property shared with $\delta$ function:

\begin{equation}
    A(\vec{r}) = 
    \int_{\Omega} A(\vec{r}^\prime) 
    W(\vec{r} - \vec{r}^\prime) \mathrm{d}\vec{r}^\prime
\end{equation}

In particle discretization, we have:

\begin{equation}
    A_i = \sum_{j} \frac{m_j}{\rho_j} A_j W_{ij}
\end{equation}

where:

\begin{equation}
    W_{ij} = W(\vec{r}_i - \vec{r}_j)\quad \text{and} 
    \quad \sum_{j} \frac{m_j}{\rho_j} W_{ij} = 1
\end{equation}

This provides a way to interpolate the value of $A$ at any point $\vec{r}$.
\end{frame}

\begin{frame}
    \frametitle{梯度的核函数插值格式}
    Consider a scalar function $A(\vec{r})$ distributed in space $\mathbb{R}^\text{dim}$. 
Let's see $\nabla A(\vec{r})$:
\begin{equation}
    \begin{aligned}
        \nabla A(\vec{r}) 
        &= \int_{\Omega} \nabla A(\vec{r}^\prime) W(\vec{r} - \vec{r}^\prime) \mathrm{d}\vec{r}^\prime \\
        &= \int_{\Omega} \nabla A(\vec{r}^\prime) W(\vec{r}^\prime - \vec{r}) \mathrm{d}\vec{r}^\prime \\
        &= \int_{\partial \Omega} A(\vec{r}^\prime) \vec{n} W(\vec{r}^\prime - \vec{r}) \mathrm{d}S^\prime
        -\int_{\Omega} A(\vec{r}^\prime) \nabla W(\vec{r}^\prime - \vec{r}) \mathrm{d}\vec{r}^\prime\\
        &= \int_{\Omega} A(\vec{r}^\prime) \nabla W(\vec{r} - \vec{r}^\prime) \mathrm{d}\vec{r}^\prime
    \end{aligned}
\end{equation}

The above formula use property of compactness and symmetry of kernel function. 
In particle discretization, we have:

\begin{equation}
    \nabla A_i = \sum_{j} \frac{m_j}{\rho_j} A_j \nabla W_{ij}
\end{equation}
\end{frame}

\begin{frame}
    \frametitle{梯度的核函数插值格式}
    where:

\begin{equation}
    \nabla W_{ij} = \nabla W(\vec{r}_i - \vec{r}_j)
\end{equation}

However, according to 'Smoothed Particle Hydrodynamic' from Monaghan 1992, 
such form of 1-st derivative interpolation is not accurate enough. 
He proposed a better way to do such interpolation. 
Supposing we have a scalar called $\phi$, let's consider $\phi\nabla A$ 's approximation:

\begin{equation}
    \begin{aligned}
        \phi\nabla A = \nabla(\phi A) - A\nabla\phi
    \end{aligned}
\end{equation}
\end{frame}

\begin{frame}
    \frametitle{梯度的核函数插值格式}

    Use interpolation above, we have:
\begin{equation}
    \begin{aligned}
        \phi_i (\nabla A)_i &= 
            \sum_j \frac{m_j}{\rho_j} \phi_j A_j \nabla W_{ij} -
            A_i\sum_j \frac{m_j}{\rho_j}  \phi_j \nabla W_{ij}\\
            &= \sum_j \frac{m_j}{\rho_j} \phi_j( A_j-A_i) \nabla W_{ij}
    \end{aligned}
\end{equation}

thus we have:

\begin{equation}
    (\nabla A)_i = \frac{1}{\phi_i}\sum_j \frac{m_j}{\rho_j} \phi_j( A_j-A_i) \nabla W_{ij} 
\end{equation}

\end{frame}

\begin{frame}
    \frametitle{梯度的核函数插值格式}

    noting that $A_{ij} = A_i - A_j$, we have:

\begin{equation}
    (\nabla A)_i = -\frac{1}{\phi_i}\sum_j \frac{m_j}{\rho_j}\phi_j A_{ij} \nabla W_{ij}
\end{equation}

For example, when we focus on $\frac{1}{\rho}\nabla p$, 
we have 2 ways to obtain its value:
\begin{equation}
    \begin{aligned}
        \phi = 1 \quad &\Rightarrow \quad 
            (\nabla p)_i = \sum_j \frac{m_j}{\rho_j} (p_j-p_i) \nabla W_{ij} \\
        \phi_i = \frac{1}{\rho} \quad &\Rightarrow \quad 
            \frac{1}{\rho}\nabla p = \nabla \left(
                \frac{p}{\rho}
            \right)+
            \frac{p}{\rho^2}\nabla \rho\\
            &\Rightarrow\quad
            \left(\frac{\nabla p}{\rho}\right)_i=
            \sum_j m_j \left(
                \frac{p_j}{\rho_j^2}+\frac{p_i}{\rho_i^2}
            \right) \nabla W_{ij}
    \end{aligned}
\end{equation}

当我推导到这里的时候,就感觉到这个数值格式似乎存在一些问题。

\end{frame}

\begin{frame}
    \frametitle{散度的核函数插值格式}
    Now let's consider a vector $\vec{A}$ distributed in space $\mathbb{R}^\text{dim}$.
    Let's see $\nabla\cdot\vec{A}$:
    \begin{equation}
        \begin{aligned}
            \nabla\cdot\vec{A}(\vec{r}) &= \int_{\Omega} \nabla\cdot\vec{A}(\vec{r}^\prime) W(\vec{r} - \vec{r}^\prime) \mathrm{d}\vec{r}^\prime \\
            &= \int_{\Omega} \nabla\cdot\vec{A}(\vec{r}^\prime) W(\vec{r}^\prime - \vec{r}) \mathrm{d}\vec{r}^\prime \\
            &= \int_{\partial \Omega} \vec{A}(\vec{r}^\prime) \cdot \vec{n} W(\vec{r}^\prime - \vec{r}) \mathrm{d}S^\prime -
            \int_{\Omega} \vec{A}(\vec{r}^\prime) \cdot \nabla W(\vec{r}^\prime - \vec{r}) \mathrm{d}\vec{r}^\prime\\
            &= \int_{\Omega} \vec{A}(\vec{r}^\prime) \cdot \nabla W(\vec{r} - \vec{r}^\prime) \mathrm{d}\vec{r}^\prime
        \end{aligned}
    \end{equation}
    
    In particle discretization, we have:
    
    \begin{equation}
        (\nabla\cdot\vec{A})_i = \sum_{j} \frac{m_j}{\rho_j} \vec{A}_j \cdot \nabla W_{ij}
    \end{equation}
    
\end{frame}

\begin{frame}
    \frametitle{散度的核函数插值格式}
    where:

\begin{equation}
    \nabla W_{ij} = \nabla W(\vec{r}_i - \vec{r}_j)
\end{equation}

However, this way o interpolation is not accurate enough. 
Let's consider a scalar $\phi$ here:

\begin{equation}
    \phi \nabla\cdot \vec{A} = \nabla\cdot(\phi \vec{A}) - \vec{A}\cdot\nabla\phi
\end{equation}

Use interpolation above, we have:

\begin{equation}
    \begin{aligned}
        \phi_i (\nabla\cdot \vec{A})_i &= 
            \sum_j \frac{m_j}{\rho_j} \phi_j \vec{A}_j \cdot \nabla W_{ij} -
            \vec{A}_i\cdot\sum_j \frac{m_j}{\rho_j}  \phi_j \nabla W_{ij}\\
            &=
            \sum_j \frac{m_j}{\rho_j} \phi_j (\vec{A}_j-\vec{A}_i) \cdot \nabla W_{ij}\\
            &=
            -\sum_j \frac{m_j}{\rho_j} \phi_j \vec{A}_{ij} \cdot \nabla W_{ij}
    \end{aligned}
\end{equation}
\end{frame}

\begin{frame}
    \frametitle{散度的核函数插值格式}
    where $\vec{A}_{ij} = \vec{A}_i - \vec{A}_j$. 
For example, when we calculate $\rho \nabla\cdot\vec{v}$, interpolation above reveals:

\begin{equation}
    (\rho \nabla\cdot \vec{v})_i = -\sum_j m_j \vec{v}_{ij} \cdot \nabla W_{ij}
\end{equation}

such interpolation is widely used in SPH method.
\end{frame}

\begin{frame}
    \frametitle{拉普拉斯算子的核函数插值格式}
    Now let's consider a scalar function $A(\vec{r})$ distributed in space $\mathbb{R}^\text{dim}$.
\begin{equation}
    \begin{aligned}
        \nabla^2 A(\vec{r}) &= \int_{\Omega} \nabla^2 A(\vec{r}^\prime) W(\vec{r} - \vec{r}^\prime) \mathrm{d}\vec{r}^\prime \\
        &= \int_{\Omega} \nabla^2 A(\vec{r}^\prime) W(\vec{r}^\prime - \vec{r}) \mathrm{d}\vec{r}^\prime \\
        &= \int_{\partial \Omega} \nabla A(\vec{r}^\prime) \cdot \vec{n} W(\vec{r}^\prime - \vec{r}) \mathrm{d}S^\prime -
        \int_{\Omega} \nabla A(\vec{r}^\prime) \cdot \nabla W(\vec{r}^\prime - \vec{r}) \mathrm{d}\vec{r}^\prime\\
        &= -\int_{\Omega} \nabla A(\vec{r}^\prime) \cdot \nabla W(\vec{r}^\prime - \vec{r}) \mathrm{d}\vec{r}^\prime\\
        &= -\int_{\partial \Omega} A(\vec{r}^\prime)\vec{n} \cdot \nabla W(\vec{r}^\prime - \vec{r}) \mathrm{d}S^\prime +
        \int_{\Omega} A(\vec{r}^\prime) \nabla^2 W(\vec{r}^\prime - \vec{r}) \mathrm{d}\vec{r}^\prime\\
        &= \int_{\Omega} A(\vec{r}^\prime) \nabla^2 W(\vec{r} - \vec{r}^\prime) \mathrm{d}\vec{r}^\prime
    \end{aligned}
\end{equation}
\end{frame}

\begin{frame}
    \frametitle{拉普拉斯算子的核函数插值格式}
    The same problem with this formula is accuracy. 
For this part, 
I consult a large number of references, 
but each reference has its own way of interpolation.

in 1985, Brookshaw proposed a way based on Taylor expansion:

\begin{equation}
    (\nabla^2 A)_i = -\sum_{j} \frac{m_j}{\rho_j} A_{ij}\frac{2|\nabla W_{ij}|}{|r_{ij}|}
\end{equation}

However, after I reading the original paper, I find he only proposed a 1-D case's formula, 
In Dan Koschier's paper, he directly use the above formula in 3-D case. 
And in another paper from Price, 2012, he use the following formula:

\begin{equation}
    (\nabla^2 A)_i = 2\sum_{j} \frac{m_j}{\rho_j} A_{ij}\frac{|\nabla W_{ij}|}{|r_{ij}|}
\end{equation}
\end{frame}

\begin{frame}
    \frametitle{拉普拉斯算子的核函数插值格式}

    My math is not good enough to prove the correctness of this formula. 
But from my personal formula, I think the following formula is more reasonable:

\begin{equation}
    \begin{aligned}
        \nabla^2 A(\vec{r})&= 
        \int_{\Omega} \nabla A(\vec{r}^\prime) \cdot \nabla W(\vec{r}- \vec{r}^\prime ) \mathrm{d}\vec{r}^\prime\\
        &\approx
        \sum_j \frac{m_j}{\rho_j} \frac{A_i-A_j}{|\vec{r}_i-\vec{r}_j|^2}
        (\vec{r}_i - \vec{r}_j)\cdot \nabla W_{ij}\\
        &=\sum_j \frac{m_j}{\rho_j}\frac{A_{ij}}{r_{ij}^2}\vec{r}_{ij}\cdot \nabla W_{ij}
    \end{aligned}
\end{equation}

From the above formula, $|\nabla W_{ij}|$ shares the same direction with $\vec{r}_{ij}$,
thus we have:

\begin{equation}
    \nabla^2 A(\vec{r})\approx
    \sum_j \frac{m_j}{\rho_j}\frac{A_{ij}}{r_{ij}}|\nabla W_{ij}|
\end{equation}

\end{frame}

\begin{frame}
    \frametitle{拉普拉斯算子的核函数插值格式}

    this result is quite similar to Price, 2012's formula except for the coefficient '$2$'.
Currently, I have no idea on why there is a '$2$' in his formula. 
A possible explanation is that it comes from the Taylor expansion of $W$.

\end{frame}

\begin{frame}
    \frametitle{向量的拉普拉斯算子的核函数插值格式}

    I copy this from Price, 2012's paper but really have no idea on how to prove it:

\begin{equation}
    \begin{aligned}
        (\nabla^2 \vec{A})_i&=-2\sum_j \frac{m_j}{\rho_j}\vec{A}_{ij}\frac{|\nabla W_{ij}|}{r_{ij}}\\
    \end{aligned}
\end{equation}

What's more, for a divergence-free vector field $\vec{A}$ (which has $\nabla\cdot\vec{A}=0$), 
Price proposed a way to calculate $\nabla^2 \vec{A}$:

\begin{equation}
    (\nabla^2 \vec{A})_i=2(d+2)\sum_{j} 
    \frac{m_j}{\rho _j} \frac{\vec{A}_{ij}\cdot \vec{r}_{ij}}{r_{ij}^2}\nabla W_{ij}
\end{equation}

For weak compressible SPH method, this scheme also works.

\end{frame}