\section{SPH方法总结}


\subsection{SPH方法思想总结}

\begin{frame}
\begin{block}{一个比喻}
    在我简单的学习和阅读文献后,
    我认为SPH方法可以用一个形象的比喻来描述:
    假想我们在流体内细密地分布一种可以吸附流体的“吸水珠”,
    可以将附近核半径内的流体吸附到自己体内,
    形成一个理想质点。

    这样做的好处是,
    在求解方程时,
    这些吸水珠同时有如下两种性质:
    \begin{itemize}
        \item 质点性质:涉及到拉格朗日描述下刚体运动求解时,
        吸水珠呈现出质点的性质,即质点的质量、速度、位置等,
        此时流体质量都集中在这个质点上;
        \item 连续介质性质:在涉及到流体运动求解时,
        需要用到速度场的空间导数,用光滑核函数把这个吸水珠打开,
        将质量、能量、动量平摊在附近的空间上,
        此时这个质点呈现出连续介质的特性。
    \end{itemize}

    在传统流体求解中,
    流体网格的粗细对求解结果精度有很大影响,
    网格越粗,精度越低,但求解越快。
    同样的,SPH方法中,
    光滑核函数半径越大,精度越低,流体分布越稀疏,但求解更快。

    而因为SPH方法中对流体质点间并没有建立明确的几何拓扑关系,
    所以这些质点往往可以自由移动,
    从而可以很好地模拟流体的自由表面,
    避免传统方法中网格畸变或者重构网格的复杂过程。
\end{block}
\end{frame}

\subsection{SPH方法优缺点总结}

\begin{frame}

    SPH方法优点:
    \begin{itemize}
        \item 适用于自由表面流动
        \item 适用于高速撞击流动,不需要网格加密
        \item 适用于多相流动,用于解决流动中的相变问题
        \item 适用于流动中的破碎问题
        \item 适用于流动中的边界大变形问题
    \end{itemize}

    SPH方法缺点:
    \begin{itemize}
        \item 计算效率低,计算量大(没有连接信息,每步都需要计算两点间距)
        \item 很难评估算法的精度与收敛性,且各类光滑核函数的选取缺乏数学理论支撑
        \item 在某些情况下会出现数值空腔,或者粒子堆积现象
        \item 其流动后处理(如计算压力、速度场等)较为困难
        \item 若要与其他方法结合,缺乏边界处物理场信息交换的方法
    \end{itemize}

\end{frame}