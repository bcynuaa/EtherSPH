\section{SPH方法简介}

\subsection{发展历程}

\begin{frame}
    \begin{itemize}
		\item 光滑粒子流体动力学(Smoothed Particle Hydrodynamics,SPH)
		方法是一种基于拉格朗日方法的流体数值模拟方法,
		由英国物理学家Lucy于1977年首次提出,Gingold和Monaghan于1977年独立提出,
		1985年Monaghan发表了SPH方法的第一篇综述性论文,SPH方法自此开始快速发展。

		\item 作为一种无网格方法(meshfree)方法,
		SPH方法并非真的不需要网格,而是求解精度不那么依赖于网格的结构。
	
		\item SPH方法通过大量粒子来离散研究对象,
	    每一个粒子代表该对象中的介质团,粒子之间无直接的网格联系,
		因此可以有效地避免传统网格方法难以处理的网格畸变问题。
	
		\item 最初的SPH方法是用于求解天体物理问题的,后来被应用于流体动力学、
		固体力学、热力学、电磁学、生物力学、地质力学等。
	\end{itemize}
\end{frame}

\begin{frame}
	SPH方法更多地还是被应用在水动力学领域
	(因其数值特性和精度问题,在气动分析中应用受到限制)。

	SPH方法有多种形式,主要有两类形式\cite{_sph_2022}:
	\begin{itemize}
		\item 不可压SPH方法(Incompressible SPH,ISPH)
		\item 弱可压SPH方法(Weakly Compressible SPH,WCSPH),
		由Monaghan于1994年引入,便于求解压力;
		而Shadloo等人验证了WCSPH方法在求解低雷诺数时翼型绕流的有效性。
	\end{itemize}

	相对而言,
	WCSPH相关理论和方法经过近20年的发展,
	计算精度已得到 明显提高,
	相比于ISPH方法,WCSPH方法涉及的GPU存储量更小,
	计算效率更高,所以更适合于开源程序的设计。
	DualSPHysics、SPHinXsys等开源程序包都是基于WCSPH理论框架的。
\end{frame}

\subsection{应用场景}

\begin{frame}
	SPH常用的应用场景有:
	\begin{itemize}
		\item 水下爆炸冲击波传播,如水下爆炸冲击波对船体的冲击;
		\item 水面波浪传播,如海洋中的海浪、风浪、涌浪等;
		\item 海岸、河流、湖泊等水体的水动力学过程,如水流泥沙、悬浮物等;
		\item 固体边界存在大变形的流固耦合问题;
		\item 生物医学领域,此时会涉及到多相流、多组分流、多物理场耦合等问题;
	\end{itemize}
\end{frame}